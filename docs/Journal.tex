% Options for packages loaded elsewhere
\PassOptionsToPackage{unicode}{hyperref}
\PassOptionsToPackage{hyphens}{url}
%
\documentclass[
]{article}
\usepackage{amsmath,amssymb}
\usepackage{lmodern}
\usepackage{ifxetex,ifluatex}
\ifnum 0\ifxetex 1\fi\ifluatex 1\fi=0 % if pdftex
  \usepackage[T1]{fontenc}
  \usepackage[utf8]{inputenc}
  \usepackage{textcomp} % provide euro and other symbols
\else % if luatex or xetex
  \usepackage{unicode-math}
  \defaultfontfeatures{Scale=MatchLowercase}
  \defaultfontfeatures[\rmfamily]{Ligatures=TeX,Scale=1}
\fi
% Use upquote if available, for straight quotes in verbatim environments
\IfFileExists{upquote.sty}{\usepackage{upquote}}{}
\IfFileExists{microtype.sty}{% use microtype if available
  \usepackage[]{microtype}
  \UseMicrotypeSet[protrusion]{basicmath} % disable protrusion for tt fonts
}{}
\makeatletter
\@ifundefined{KOMAClassName}{% if non-KOMA class
  \IfFileExists{parskip.sty}{%
    \usepackage{parskip}
  }{% else
    \setlength{\parindent}{0pt}
    \setlength{\parskip}{6pt plus 2pt minus 1pt}}
}{% if KOMA class
  \KOMAoptions{parskip=half}}
\makeatother
\usepackage{xcolor}
\IfFileExists{xurl.sty}{\usepackage{xurl}}{} % add URL line breaks if available
\IfFileExists{bookmark.sty}{\usepackage{bookmark}}{\usepackage{hyperref}}
\hypersetup{
  pdftitle={Analysis journal},
  pdfauthor={Jay Gillenwater},
  hidelinks,
  pdfcreator={LaTeX via pandoc}}
\urlstyle{same} % disable monospaced font for URLs
\usepackage[margin=1in]{geometry}
\usepackage{graphicx}
\makeatletter
\def\maxwidth{\ifdim\Gin@nat@width>\linewidth\linewidth\else\Gin@nat@width\fi}
\def\maxheight{\ifdim\Gin@nat@height>\textheight\textheight\else\Gin@nat@height\fi}
\makeatother
% Scale images if necessary, so that they will not overflow the page
% margins by default, and it is still possible to overwrite the defaults
% using explicit options in \includegraphics[width, height, ...]{}
\setkeys{Gin}{width=\maxwidth,height=\maxheight,keepaspectratio}
% Set default figure placement to htbp
\makeatletter
\def\fps@figure{htbp}
\makeatother
\setlength{\emergencystretch}{3em} % prevent overfull lines
\providecommand{\tightlist}{%
  \setlength{\itemsep}{0pt}\setlength{\parskip}{0pt}}
\setcounter{secnumdepth}{-\maxdimen} % remove section numbering
\ifluatex
  \usepackage{selnolig}  % disable illegal ligatures
\fi

\title{Analysis journal}
\author{Jay Gillenwater}
\date{8/25/2021}

\begin{document}
\maketitle

\hypertarget{overview}{%
\subsection{Overview}\label{overview}}

This document will be a day-by-day record of what I do in the
development of this workflow. I'm making this a few days into
development though so I've grouped the first few days into one section.

\hypertarget{before-2021-08-25}{%
\subsubsection{Before 2021-08-25}\label{before-2021-08-25}}

Set up the workflow with tflow and targets. Got some functions working
to read in and clean up the phenotype data from the excel spreadsheet.
Also worked on some functions for some simple exploratory visualizations
(boxplots, histograms, and QQ plots). I used these plots to find one
sample that was likely an error (LA PLOT: 255) and removed it from the
data. I also added a function that takes the clean data with the likely
bad sample removed and calculates average values for each phenotype for
each genotype. The function also calculates lsmeans but the average may
be more appropriate to use since many of the genotypes were not balanced
across locations. I still want to go back and compare the outputs from
this function/look into what version would be best to use as the
phenotypes for mapping. Maybe change it to also give by location
averages for each phenotype in addition to the average across location.

Yesterday I worked on cleaning up the genotype data into a format that
works with r/qtl. I settled on the ``csvsr'' format since the export
from genomestudio was already close to this format and the phenotype
data is already separate. The main challenge was converting the genotype
data into the ``ABH'' format expected by r/qtl. The NC-Raleigh parent
(code 2104) was genotyped, but the soja parent was not so I was not able
to directly convert the observed progeny genotypes to the ABH format
based on the observed founder genotypes. One approach I tried was to use
the genotypes provided when the USDA germplasm collection was genotyped
with the 50k SNP chip. The 6K SNP chip that was used to genotype this
population uses a subset of the markers from the 50K chip and as such,
the historical genotypes could also be subset to the markers used to
genotype the progeny. I tried this approach but the alleles present in
the historical sample did not match up with those observed to be
segregating in the current population. As an example, the Raleigh parent
had a genotype of AA at a SNP, and the historical Soja genotype had a
genotype of GG, but the observed genotypes across samples in the current
data were only AA, AC, and CC. Because of this mismatch, I instead opted
to ``derive'' the soja parents genotype at each marker by observing what
alleles were present at each site. Using the same example from above, I
would assume that the founder parent had a genotype of CC at that
particular SNP. Each SNP in the data only had two alleles, so this
approach worked although I wasn't able to so some quality control steps
like removing SNPs that were missing or heterozygous in the soja parent.

I want to revisit using the historical data though. I expected more of
the alleles to fit with the observed segregation patterns just trough
chance alone than I saw when I joined the historical soja genotypes to
the new data. I was worried that there could be a problem with how I
joined the data with the SNP names, but I haven't been able to find an
error in that part (although I'm still looking). Maybe something in the
conversion of the dbSNP names in the historical data to the longer names
that are used in the 6K chip? Differences in the way the genotypes are
presented in the two sources may also be causing the mismatch, I have to
look into the technical details of the two chips to be sure though.

Added functions to export the cleaned phenotype and genotype data to
external files that are ready to be input into r/qtl and another
function to actually read them into r/qtl.

\hypertarget{todo}{%
\paragraph{TODO}\label{todo}}

Figure out why the alleles in the historical SNP data for the soja
parent don't match those in the current data.

\hypertarget{section}{%
\subsubsection{2021-08-25}\label{section}}

Explored some basic quality checks for linkage mapping with the cross
data by using ASMap. Really just exploring the data and seeing how many
markers pass different thresholds for missingness and segregation
distortion.

The data does seem to have more heterozygosity than I expected, about
8.5\% before any filtering. I'll have to check if this is due to a few
problematic SNPs or is more general. I think I should go back to the
starting genotype matrix as it was exported from genomestudio and get
some summary stats for each marker and the data as a whole to get a
better sense of the magnitude and distribution of the heterozygosity.

I ran the mapping function (mstmap.cross) on the data without much
filtering too, just to see what I'd get out really. In general, the
marker orders math well with their physical orders so that's good. The
genetic lengths of the linkage groups are much larger than what I'd
expect from a final map (going off past maps using the same SNP chip at
least). I haven't done much filtering of the SNPs or genotypes yet so
that wasn't unexpected. Next I want to work on cleaning up the map.
Right now, I think the best way to do this is still to use the ASMap
functions interactively, I'll have to be careful as I go along though to
keep everything organized because ultimately, I want to repeat the steps
to get the final map in a function that I can use in the overall
workflow.

\end{document}
