% Options for packages loaded elsewhere
\PassOptionsToPackage{unicode}{hyperref}
\PassOptionsToPackage{hyphens}{url}
%
\documentclass[
]{article}
\usepackage{amsmath,amssymb}
\usepackage{lmodern}
\usepackage{ifxetex,ifluatex}
\ifnum 0\ifxetex 1\fi\ifluatex 1\fi=0 % if pdftex
  \usepackage[T1]{fontenc}
  \usepackage[utf8]{inputenc}
  \usepackage{textcomp} % provide euro and other symbols
\else % if luatex or xetex
  \usepackage{unicode-math}
  \defaultfontfeatures{Scale=MatchLowercase}
  \defaultfontfeatures[\rmfamily]{Ligatures=TeX,Scale=1}
\fi
% Use upquote if available, for straight quotes in verbatim environments
\IfFileExists{upquote.sty}{\usepackage{upquote}}{}
\IfFileExists{microtype.sty}{% use microtype if available
  \usepackage[]{microtype}
  \UseMicrotypeSet[protrusion]{basicmath} % disable protrusion for tt fonts
}{}
\makeatletter
\@ifundefined{KOMAClassName}{% if non-KOMA class
  \IfFileExists{parskip.sty}{%
    \usepackage{parskip}
  }{% else
    \setlength{\parindent}{0pt}
    \setlength{\parskip}{6pt plus 2pt minus 1pt}}
}{% if KOMA class
  \KOMAoptions{parskip=half}}
\makeatother
\usepackage{xcolor}
\IfFileExists{xurl.sty}{\usepackage{xurl}}{} % add URL line breaks if available
\IfFileExists{bookmark.sty}{\usepackage{bookmark}}{\usepackage{hyperref}}
\hypersetup{
  pdftitle={Progress Overview},
  pdfauthor={Jay Gillenwater},
  hidelinks,
  pdfcreator={LaTeX via pandoc}}
\urlstyle{same} % disable monospaced font for URLs
\usepackage[margin=1in]{geometry}
\usepackage{graphicx}
\makeatletter
\def\maxwidth{\ifdim\Gin@nat@width>\linewidth\linewidth\else\Gin@nat@width\fi}
\def\maxheight{\ifdim\Gin@nat@height>\textheight\textheight\else\Gin@nat@height\fi}
\makeatother
% Scale images if necessary, so that they will not overflow the page
% margins by default, and it is still possible to overwrite the defaults
% using explicit options in \includegraphics[width, height, ...]{}
\setkeys{Gin}{width=\maxwidth,height=\maxheight,keepaspectratio}
% Set default figure placement to htbp
\makeatletter
\def\fps@figure{htbp}
\makeatother
\setlength{\emergencystretch}{3em} % prevent overfull lines
\providecommand{\tightlist}{%
  \setlength{\itemsep}{0pt}\setlength{\parskip}{0pt}}
\setcounter{secnumdepth}{-\maxdimen} % remove section numbering
\ifluatex
  \usepackage{selnolig}  % disable illegal ligatures
\fi

\title{Progress Overview}
\author{Jay Gillenwater}
\date{1/08/2022}

\begin{document}
\maketitle

\hypertarget{introduction}{%
\section{Introduction}\label{introduction}}

This document is a summary of the analysis that has already been done
and the analysis that is planned for the Raleigh x Soja mapping
population.

\hypertarget{progress-so-far}{%
\section{Progress so far}\label{progress-so-far}}

\hypertarget{data-descriptions}{%
\subsection{Data descriptions}\label{data-descriptions}}

\hypertarget{population-structure}{%
\subsubsection{Population structure}\label{population-structure}}

The mapping population consists of 151 recombinant inbred line (RIL)
soybean genotypes in the \(F_4\) generation. Genotypes are identified
through numeric codes that range from 1901 to 2105. Codes 2104 and 2105
are the NC Raleigh and PI 424025B (Soja) population parents
respectively.

\hypertarget{phenotypic-distributions}{%
\subsubsection{Phenotypic
distributions}\label{phenotypic-distributions}}

\newpage

\begin{verbatim}
## `summarise()` has grouped output by 'code', 'loc'. You can override using the `.groups` argument.
\end{verbatim}

\begin{verbatim}
## Warning: Using `bins = 30` by default. Pick better value with the argument
## `bins`.
\end{verbatim}

\begin{verbatim}
## Warning: `cols` is now required when using unnest().
## Please use `cols = c(lab.data)`
\end{verbatim}

\begin{verbatim}
## Warning: Using `bins = 30` by default. Pick better value with the argument
## `bins`.
\end{verbatim}

\begin{verbatim}
## Warning: `cols` is now required when using unnest().
## Please use `cols = c(lab.data)`
\end{verbatim}

\begin{verbatim}
## Warning: Using `bins = 30` by default. Pick better value with the argument
## `bins`.
\end{verbatim}

\begin{verbatim}
## Warning: `cols` is now required when using unnest().
## Please use `cols = c(lab.data)`
\end{verbatim}

\includegraphics{progress_report_files/figure-latex/phenoHistograms-1.pdf}

\hypertarget{genotypic-data-summary}{%
\subsubsection{Genotypic data summary}\label{genotypic-data-summary}}

\hypertarget{linkage-mapping}{%
\subsection{Linkage mapping}\label{linkage-mapping}}

\hypertarget{data-cleaning-procedures}{%
\subsubsection{Data cleaning
procedures}\label{data-cleaning-procedures}}

\hypertarget{phenotypes}{%
\paragraph{Phenotypes}\label{phenotypes}}

\hypertarget{genotypes}{%
\paragraph{Genotypes}\label{genotypes}}

\hypertarget{final-map-statistics}{%
\subsubsection{Final map statistics}\label{final-map-statistics}}

\hypertarget{planned-analyses}{%
\section{Planned Analyses}\label{planned-analyses}}

\hypertarget{mapping}{%
\subsection{Mapping}\label{mapping}}

\hypertarget{summaries}{%
\subsection{Summaries}\label{summaries}}

\end{document}
